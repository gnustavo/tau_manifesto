\documentclass{article}
\usepackage{fancyvrb}
\usepackage{xcolor}
\usepackage{pygments}

\usepackage{polytexnic}
\begin{document}

\title{The Tau Manifesto}
\author{Michael Hartl}
\date{Tau Day, 2010}
\maketitle

\section{The most important number in mathematics} % (fold)
\label{sec:the_most_important_number_in_mathematics}

Books have been written. I mean, \emph{books!} Some poor dudes memorize dozens, even hundreds or thousands of its digits. What kind of sad sack memorizes fifty digits of $\pi$ (\hyperref[fig:michael_hartl]{Figure~}\ref{fig:michael_hartl})?

\begin{figure}
\begin{center}
\image{images/figures/michael_hartl.jpg}
\end{center}
\caption{This poor bastard memorized pi to 50 decimal places.\label{fig:michael_hartl}}
\end{figure}

As mathematician Bob Palais observed in his delightful article ``$\pi$ is Wrong!'',\footnote{Palais, Robert. ``$\pi$ is Wrong!'', \emph{The Mathematical Intelligencer}, Volume~23, Number~3, 2001, pp.~7--8.} $\pi$ is wrong. As he argued, and as this manifesto will prove, the correct number is actually $2\pi$. But even the plucky Professor Palais couldn't quite bring himself to take the idea fully seriously, introducing a weird three-legged symbol for $2\pi$ (Fig), which he called ``one turn''. As we'll see, the description is prescient, but the symbol, lamentably, is D.O.A. 

The $\tau$ manifesto is dedicated to the proposition that when someone says ``$\pi$ is wrong'', the proper response is ``No, \emph{really}.'' And the correct number, tragically known as $2\pi$, deserves a proper name. The $\tau$ manifesto proposes that this name---as I hope you will not be surprised to learn---should be~$\tau$.

  \subsection{Why tau?} % (fold)
  \label{sec:why_tau}
  
  % subsection why_tau (end)

\begin{enumerate}
  \item \textbf{No serious conflicts.} $\tau$ has no serious conflicts with current usage.\footnote{In programming terms, there are no serious namespace clashes.} The symbol~$\tau$ is used for, e.g., \emph{strain} in mechanical engineering and \emph{proper time} in special and general relativity, but there is no universal conflicting usage.\footnote{Minor clashes are OK; after all, physicists manage to use $e$ both for the natural number and for the charge on an electron without causing apparent harm.} 
  
  \item \textbf{Physical resemblance.} $\tau$ is typographically similar to $\pi$, thereby evoking the proper image of a circle constant. (Unfortunately, the number of ``legs'' isn't quite right: it would be poetic if we could write $\pi = 2\tau$, but it wasn't meant to be.)
  
  \item \textbf{Etymology.} In the context of radian angle measure, $\tau$ represents one \emph{turn} of a circle. The root of the English word ``turn'' is the Greek word for ``lathe'', \emph{tornos}---or, as the Greeks would put it: 
  
\[ \tau \acute{o}\rho\nu o\varsigma \]
  
\end{enumerate}

Looking at the first letter of that Greek lathe, I'm going to go out on a limb here and say: \href{http://en.wikipedia.org/wiki/Q.E.D.}{\emph{quod erat demonstrandum}}.

% section the_most_important_number_in_mathematics (end)

\section{The nature of the circle} % (fold)
\label{sec:the_nature_of_the_circle}

Long considered the perfect geometric form, the circle is defined as the only shape with constant diameter. Wait---no, it's not. There are infinitely many shapes with constant diameter (Fig). A circle is defined as the set of all points equidistant from a given point. That point is the \emph{center} of the circle, and the distance is the \emph{radius}. There is not a diameter to be

% section the_nature_of_the_circle (end)

\section{Radian angle measure} % (fold)
\label{sec:radian_angle_measure}

Lorem ipsum dolor sit amet, consectetur adipisicing elit, sed do eiusmod tempor incididunt ut labore et dolore magna aliqua. Ut enim ad minim veniam, quis nostrud exercitation ullamco laboris nisi ut aliquip ex ea commodo consequat. Duis aute irure dolor in reprehenderit in voluptate velit esse cillum dolore eu fugiat nulla pariatur. Excepteur sint occaecat cupidatat non proident, sunt in culpa qui officia deserunt mollit anim id est laborum.

  \subsection{The circle functions} % (fold)
  \label{sec:the_circle_functions}
  
Lorem ipsum dolor sit amet, consectetur adipisicing elit, sed do eiusmod tempor incididunt ut labore et dolore magna aliqua. Ut enim ad minim veniam, quis nostrud exercitation ullamco laboris nisi ut aliquip ex ea commodo consequat. Duis aute irure dolor in reprehenderit in voluptate velit esse cillum dolore eu fugiat nulla pariatur. Excepteur sint occaecat cupidatat non proident, sunt in culpa qui officia deserunt mollit anim id est laborum.
  
  % subsection the_circle_functions (end)

% section radian_angle_measure (end)

\[ e^{i\tau} = 1 \]


This is $\tau$. It is the true circle constant. After seeing how awesome $\tau$ is, $\pi$ will seem lame and old-fashioned.

\section{Puns}

We come now to the final objection: ``What about puns?'', I hear you cry. I know, I know, ``$\pi$ in the sky'' is so very clever. And yet, $\tau$ itself is pregnant with possibilities. After all, once you have accepted $\tau$ism, you will become a $\tau$ist like me. It is not $\tau$ that is a piece of $\pi$, but $\pi$ that is a piece of $\tau$---one-half~$\tau$, to be exact. This is the true nature of the~$\tau$.

About the author

Michael Hartl is an educator and entrepreneur. He is the founder of the \href{http://www.railstutorial.org/}{Ruby on Rails Tutorial project}, which teaches \href{http://rubyonrails.org/}{Ruby on Rails} web development through the \href{http://www.railstutorial.org/book}{\emph{Ruby on Rails Tutorial} book} and \href{http://www.railstutorial.org/}{tutorial screencasts}. Previously, he taught theoretical and computational physics at the \href{http://www.caltech.edu/}{California Institute of Technology} (Caltech) for six years, where he received the Lifetime Achievement Award for Excellence in Teaching in 2000. Michael is a graduate of \href{http://college.harvard.edu/}{Harvard College}, has a \href{http://resolver.caltech.edu/CaltechETD:etd-05222003-161626}{Ph.D. in Physics} from \href{http://www.caltech.edu/}{Caltech}, and is an alumnus of the \href{http://ycombinator.com/}{Y~Combinator} program.

Michael knows 51 digits of $\pi$, i.e., fifty decimal places---approximately 48 more than Matt Groening. He is currently working on memorizing 52 digits of $\tau$.

\end{document}

